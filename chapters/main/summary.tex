% \chapter*{Tóm tắt khóa luận}
\chapter*{\centering\Large{Tóm tắt khóa luận}}
\addcontentsline{toc}{chapter}{Tóm tắt khóa luận}

Vào năm 2008, Lizhe Wang, một trong những ngọn cờ tiên phong đã đưa ra khái
niệm cơ bản về “Cloud Computing”. Theo đó, điện toán đám mây là một tập hợp các
dịch vụ hỗ trợ mạng, cung cấp khả năng mở rộng, chất lượng dịch vụ được đảm bảo,
thường được cá nhân hóa, chi phí thấp theo yêu cầu và có thể truy cập một cách đơn giản
và phổ biến \cite{wang2008scientific}.\\

Từ thời điểm đó, điện toán đám mây đã dần trở thành một trong những từ khóa
tiếp theo của ngành công nghệ thông tin. Vào năm 2020, tổng giá trị của thị trường là
371,4 tỷ USD, tốc độ tăng trưởng kép hàng năm (CAGR) là 17,5\%. Theo dự kiến, tới
năm 2025, thị trường điện toán đám mây sẽ có giá trị tới 832,1 tỷ USD, sẽ có hơn 100
zettabytes dữ liệu được lưu trữ trên đám mây. Trong cùng khoảng thời gian đó, tổng
dung lượng dữ liệu được lưu trữ sẽ vượt quá 200 zettabytes, nghĩa là có khoảng 50\% dữ
liệu được lưu trên đám mây, con số này là 25\% vào năm 2015 \cite{vladimir2021cloudcomputing}.\\

Ở Việt Nam hiện nay, đón đầu xu thế trong cuộc cách mạng công nghệ 4.0, Chính
phủ thúc đẩy mạnh mẽ và tạo điều kiện trên mọi phương diện nhằm thúc đẩy khởi nghiệp,
nhóm ngành công nghệ thông tin là một trong những lĩnh vực được ưu tiên thúc đẩy phát
triển. Nhiều công ty công nghệ thông tin được thành lập và có tiềm năng phát triển cao.
Đi đôi với việc thành lập doanh nghiệp, các công ty xây dựng cơ sở dữ liệu của mình
nhằm phục vụ các hoạt động nội bộ của công ty cũng như kinh doanh với đối tác, khách
hàng. Xu hướng sử dụng cloud service để lưu và quản trị cơ sở dữ liệu được nhiều đơn
vị lựa chọn.\\

Tuy có những ưu điểm vượt trội như trên, hệ thống này cũng đặt ra nhiều vấn đề
về an toàn thông tin cho dữ liệu được lưu trữ trên đám mây. Môi trường đám mây được
xem là không tin cậy cho những dữ liệu riêng tư có tính nhạy cảm: dữ liệu có thể bị truy
cập trái phép từ nhà quản cung cấp dịch vụ, bị rò rỉ sang bên thứ ba không liên quan hoặc
bị đánh cắp bởi quản trị viên, hacker. Các công ty cho thuê dịch vụ đám mây phục vụ
nhiều khách hàng có thể dẫn tới việc không duy trì được sự tách biệt giữa những người cùng thuê dịch vụ. Dữ liệu của người dùng sau khi hết sử dụng có thể không được xóa an toàn do bị giảm khả năng hiển thị vào nơi dữ liệu được lưu trữ vật lý trên đám mây hoặc bị mất mát do nhà cung cấp dịch vụ vô tình xóa hoặc thảm họa. Người dùng nội bộ của công ty cho thuê có thể lạm quyền và truy cập nhằm đánh cắp thông tin \cite{timothy2018risk}. Do vậy, ngoài những cơ chế bảo mật sẵn có từ nhà cung cấp dịch vụ, chủ sở hữu dữ liệu cần có những cơ chế bảo mật riêng để đảm bảo thông tin lưu trữ, tương tác và trao đổi được an toàn. Một trong những giải pháp được sử dụng là mã hóa dữ liệu trước khi lưu trữ tại đám mây và thay đổi phương pháp tương tác truyền thống trên dữ liệu rõ bằng các phương pháp tương tác trên dữ liệu mã hóa. \\ 

Để thử nghiệm và tìm ra phương pháp mã hóa, kiểm soát truy cập với cơ sở dữ liệu được lưu trữ trên đám mây, nhóm tác giả đã tiến hành nghiên cứu và xây dựng mô hình nhằm mục đích đảm bảo sự an toàn của cơ sở dữ liệu trước những rủi ro của hệ thống điện toán đám mây. Từ đó, rút ra nhận xét và hướng phát triển trong tương lai. \\

Phần còn lại của khóa luận được chia thành 5 chương. Chương 1 giới thiệu những nét khái quát chung và những vấn đề cơ bản đặt ra khiến nghiên cứu này là cần thiết. Chương 2 và chương 3 trình bày một số nghiên cứu hiện tại về kiểm soát truy cập và mã hóa dữ liệu, cũng như đưa ra một số chi tiết về phương pháp của nhóm tác giả cho vấn đề được nêu ra ở chương 1. Chương 4 trình bày mô tả ứng dụng hiện thực về kiểm soát truy cập dựa trên mã hóa CP-ABE. Chương 5 tổng kết lại một số kết quả đạt được và hướng phát triển của nghiên cứu.

